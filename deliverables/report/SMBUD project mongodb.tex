%----------------------------------------------------------------------------------------
%	PACKETS AND CONFIGURATION
%----------------------------------------------------------------------------------------

\documentclass[12pt, a4paper]{article}

\usepackage{times} % Times New Roman font
\usepackage{graphicx} % 'graphics' package interface
\usepackage{geometry} % Edit document margins
\usepackage{hyperref} % Table of contents hyperlinks
\usepackage{tcolorbox} % Colored boxes for code
\usepackage[font=small, labelfont=bf]{caption} % Image caption font
\usepackage{longtable} % Build long tables

% HELPER PACKAGES (REMOVE IN FINAL) %
\usepackage{blindtext} % Lorem Ipsum
\usepackage{todonotes} % TODOs as useful reminders
\setlength{\marginparwidth}{2cm} % Otherwise todonotes gets angry at me lol
\setuptodonotes{fancyline, color=green!40, shadow} % TODOs options
% HELPER PACKAGES (REMOVE IN FINAL) %

\graphicspath{ {./res/} } % Path to graphics
\hypersetup{    % ToC Hyperlink setup
    colorlinks,
    citecolor=blue,
    filecolor=blue,
    linkcolor=blue,
    urlcolor=blue
}

%----------------------------------------------------------------------------------------
%	DOCUMENT
%----------------------------------------------------------------------------------------

\begin{document}

\newgeometry{top=7cm, bottom=2cm} % Setting the margins for the title

% Title
\begin{titlepage}
    \centering
    {\Huge\bfseries Pandemic Information System Model\par} % Project title
    \vspace{1.5cm}
    {\scshape\large Systems and Methods for Big and Unstructured Data \par} % Course
    \vspace{0.5cm}
    {\scshape\large Prof. Marco Brambilla \par} % Professor
    \vspace{1cm}
    {\scshape\large % Description
        Second delivery \par 
        MongoDB Project \par 
    }
    \vspace{0.5cm}
    {\slshape\large December 2021 \par} % Date
    \vspace{1cm}
    \linespread{0.8} % Authors interline
    {\large\itshape % Authors
        Avci Oguzhan - \texttt{10557284}\\
        Gentile Nicole - \texttt{10594355}\\
        Rigamonti Davide - \texttt{10629791}\\
        Singh Raul - \texttt{10623232}\\
        Tagliaferri Mattia - \texttt{10572418}
    }
    \vfill
    \begin{figure}[b]
        \includegraphics[scale=0.6]{polimi.png} % Polimi logo
        \centering
    \end{figure}

    \pagenumbering{gobble} % Remove page number

\end{titlepage}

\newgeometry{bottom=3cm} % Reset the margins
\pagenumbering{arabic} % Reset the page number

\clearpage

% INDEX
{
    \hypersetup{hidelinks}
    \tableofcontents
}

% LIST OF TODOs (REMOVE IN FINAL)
\listoftodos

\clearpage

% INTRODUCTION
\section{Introduction}

\subsection{Problem Specification}

The idea of the project is that of building a dataset containing information about 
\emph{"green"} certifications and people/organizations acquiring and releasing them. \\ 
A certificate is released when a person takes a vaccination shot or takes a test to check
if they are infected with Covid-19, all certificates contain information about the person 
involved, the organization issuing the certificate and information about the vaccine or 
the test. \\
The purpose of this project is to build a system that allows to check the validity of 
these certificates according to the latest government rules.

\subsection{Hypoteses}

\begin{itemize}

    \item Vaccines and tests
    \begin{itemize}
        \item[] We assumed that vaccines and tests don't expire and can be used 
            as soon as they are produced without accounting for delivery time.
        \item[] Similarly to what happens in reality, we assumed that there is one
            certification for each test/vaccine, so that a person can have different 
            certifications associated but not all of them may be valid at a given time.
        \item[] We also assumed that the minimum time elapsed between two doses
            for the same person is at least one day in order to have a better
            distribution of vaccinations inside the generated dataset;
            furthermore, all doses must be of the same vaccine brand and a
            single person can have up to 3 vaccine doses. 
        \item[] Given the previous assumptions we don't consider the test 
            results to have any relevance besides for negative tests
            generating a certification that is valid for a certain amount of 
            time.
        \item[] Vaccines are produced from 01/01/2021 to 26/11/2021 with a
            number ranging from 0 to 2 lots per day, adding up to a total
            of 330 lots numbered from \emph{1} to \emph{330}.
        \item[] Tests are performed from 01/01/2021 to 26/11/20 21 and 
            their lots are numbered from \emph{1200} to \emph{1600}.
    \end{itemize}

    \item Other entities
    \begin{itemize}
        \item[] The SSN field for people is considered to be a unique
            identifier for the single individual.
        \item[] Names for regular people, doctors, cities and authorized bodies 
            have no relevance and were chosen randomly, therefore information 
            such as phone numbers for people and coordinates for authorized 
            bodies aren't designed to be consistent among the dataset.
        \item[] All authorized bodies can release all kinds of certificates.
        \item[] People were born between years 1941 and 2011, therefore their 
            age ranges roughly from \emph{10 y.} to \emph{80 y.} considering 
            2021 to be current year.
    \end{itemize}

\end{itemize}
  
\clearpage

% DATABASE
\section{Database}

\subsection{Document Diagram}

\todo{Do-cument Diagram}
\blindtext

\subsection{Dataset description}

We represented a population of 200 people taking a total of 330 vaccinations
(190 first doses, 120 second doses and 20 third doses) and 300 tests inside
a total of 270 authorized bodies (130 hospitals, 80 pharmacies and 60 
specialized centers). \\
Certifications are characterized by information about the related person
(SSN, name, surname, birthdate, city, phone number and an emergency contact
including the name, surname, phone number and description for a person to 
contact in case of emergency) and the vaccination (date, organization, doctor 
and information about the vaccine such as brand, type, lot and production 
date) or the test (lot, result, date, organization and doctor). \\
There is also a document containing informations about the latest government rules
on the time validity of tests and vaccines certifications; we assumed test validity is 
24 hours (in case a negative test implies that the person has healed, its duration is 
extended to 6 months), while vaccinations validity is 9 months. \\
The complete script used for the creation of the dataset, togheter with a 
database dump will be provided alongside this document.

\subsection{Queries}

\todo{Queries}
\blindtext

\subsection{Commands}

\todo{Com-mands}
\blindtext

\clearpage

% APPLICATION
\section{Application}

\subsection{Description}

\todo{De-scription}
\blindtext

\subsection{User Guide}

\todo{User Guide}
\blindtext

\clearpage % Here otherwise the screenshots go in strange places

\subsection{Screenshots}

\todo{Screen-shots}
\blindtext

\clearpage 

% REFERENCES AND SOURCES
\section{References and sources}

In order to develop these project, the following tools were used:

\begin{itemize}
    \item MongoDB and the MongoDB Compass to build and navigate the database 
        and write queries;
    \item \LaTeX~to write the report;
    \item Github as a versioning and collaboration mean;
    \item \url{https://www.Mockaroo.com} 
        to generate realistic data for people, authorized bodies and 
        certifications;
    \item \href{https://data.gov.sg/dataset/listing-of-licensed-pharmacies}{link}
        to the resource used for the pharmacy population;
    \item \href{https://corgis-edu.github.io/corgis/csv/hospitals/}{link}
        to the resource used for the hospital population.
\end{itemize}

\clearpage

\end{document}